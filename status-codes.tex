\documentclass[final,table]{beamer}

\usecolortheme{dove}

\usepackage[orientation=portrait,size=a1]{beamerposter}
\usepackage{graphicx}

\definecolor{light-gray}{gray}{0.95}

\setlength{\tabcolsep}{18pt}
\renewcommand{\arraystretch}{1.1}

\input{./footer.tex}

\begin{document}
  \begin{frame}{}

    \begin{minipage}{\textwidth}
      \centering
      \includegraphics[]{./title-status-codes.pdf}
    \end{minipage}

    \vspace{0.5in}

    \begin{columns}
      \begin{column}{0.05\textwidth}
      \end{column}
      \begin{column}{0.9\textwidth}
        \begin{block}{}
          \large
            HTTP status codes are returned in the response. They each consist of
            a three digit numerical code and a text description (the text
            description is simply advisory, and may be translated to other
            languages). The codes are categorized such that general classes of
            errors have the same most-significant digit, so if a client does not
            recognize a status code it may still infer the code's category.
          \normalsize
        \end{block}
      \end{column}
      \begin{column}{0.05\textwidth}
      \end{column}
    \end{columns}

    \vspace{0.5in}

    \begin{block}{\huge 1XX Informational}

      \vspace{0.3in}

      \rowcolors{3}{light-gray}{white}
      \begin{tabular}{p{0.3\textwidth} p{0.65\textwidth}}
        Status Code & Description \\ \hline
        100 Continue & The server has received the request headers, and the client should begin to send the request body \\
        101 Switching Protocols & The server has received a request to switch protocols and is doing so \\
        102 Processing (WebDAV) & Indicates that the server has received the request, used by WebDAV to avoid timeouts for long-running requests \\
      \end{tabular}
    \end{block}
    \begin{block}{\huge 2XX Success}

      \vspace{0.3in}

      \rowcolors{3}{light-gray}{white}
      \begin{tabular}{p{0.3\textwidth} p{0.65\textwidth}}
        Status Code & Description \\ \hline
        200 OK & Standard response for successful HTTP requests \\
        201 Created & The request was successful, \emph{and} a new resource has been created \\
        202 Accepted & The request was accepted for processing, but the job hasn't actually been completed. It is possible that the request will be rejected once processing takes place \\
        203 Non-Authoritative Information & The request was successfully processed, but the returned information may be from an untrusted third party \\
        204 No Content & The request was processed and no content is being returned \\
        206 Partial Content & Only part of the request body is being delivered (for example, when resuming an interrupted download) \\
        207 Multi-Status (WebDAV) & The message body is an XML document and may contain multiple status codes per sub-requests \\
        208 Already Reported (WebDAV) & The response has already been enumerated in a previous reply and will not be reported again\\
      \end{tabular}
    \end{block}
    \begin{block}{\huge 3XX Redirection}

      \vspace{0.3in}

      \rowcolors{3}{light-gray}{white}
      \begin{tabular}{p{0.3\textwidth} p{0.65\textwidth}}
        Status Code & Description \\ \hline
        300 Multiple Choices & Indicates that there are multiple locations which the client may follow \\
        301 Moved Permanently & The resource has moved permanently, and all future requests should use the give URL instead \\
        302 Found & The resource has been found (or moved temporarily). The HTTP/1.0 specification requires that the redirect uses the same verb, but in practice clients use a GET as in a 303. See 303 and 307 \\
        303 See Other & The resource has been found, and should be accessed using a GET method. Added in HTTP/1.1 to clarify the ambiguity in the behavior of status 302. See 302 and 307 \\
        304 Not Modified & The resource has not been modified since the last time the client has cached it \\
        305 Use Proxy & The resource should be accessed through a specified proxy \\
        307 Temporary Redirect & The request should be repeated with the same request method at the given address. Added in HTTP/1.1 to clarify the ambiguity in the behavior of status 302. See 302 and 303
      \end{tabular}
    \end{block}
    \begin{block}{\huge 4XX Client Error}

      \vspace{0.3in}

      \rowcolors{3}{light-gray}{white}
      \begin{tabular}{p{0.3\textwidth} p{0.65\textwidth}}
        Status Code & Description \\ \hline
        400 Bad Request & The request can not be fulfilled because the request contained bad syntax \\
        401 Unauthorized & The client needs to authenticate in order to access this resource \\
        402 Payment Required & This code is intended to be used for a micropayment system, but the specifics for this system are unspecified and this code is rarely used \\
        403 Forbidden & The client is not allowed to access this resource. Generally, the client is authenticated and does not have sufficient permission \\
        404 Not Found & The resource was not found, though its existence in the future is possible \\
        405 Method Not Allowed & The method used in the request is not supported by the resource \\
        406 Not Acceptable & The server can not generate content which is acceptable to the client according to the request's ``Accept'' header \\
        407 Proxy Authentication Required & The client must authenticate with the proxy \\
        408 Request Timeout & The client did not complete its request in a reasonable timeframe \\
        409 Conflict & The request could not be completed due to a conflict in state (for example, attempting to update a resource when it has changed since last access) \\
        410 Gone & The resource is gone, and will always be gone; the client should not request the resource again \\
        411 Length Required & The request is missing its ``Content-Length'' header, which is required by this resource \\
        412 Precondition Failed & The server can not meet preconditions specified in the client request \\
        413 Request Entity Too Large & The request body is larger than the server will process \\
        414 Request-URI Too Long & The request URI is too long for the server to process \\
        415 Unsupported Media Type & The server can not process the request body because it is of an unsupported MIME type \\
        416 Requested Range Not Satisfiable & The client has asked for portion of a file that the server can not supply (ie, a range of bytes outside the size of the requested file) \\
        417 Expectation Failed & The server can not meet the requirements of the ``Expect'' header in the request \\
        418 I'm a teapot (HTCPCP) & Returned by teapots implementing the HyperText Coffee Pot Control Protocol \\
        420 Enhance Your Calm (Twitter) & The client is being rate-limited; a reference to cannabis culture \\
        422 Unprocessable Entity (WebDAV) & The server can not process the request due to semantic errors \\
        423 Locked (WebDAV) & The resource is currently locked \\
        424 Failed Dependency (WebDAV) & The request failed because of a previously-failed request \\
        429 Too Many Requests & The client is being rate-limited \\
        431 Request Header Fields Too Large & Either a single request header is too large, or all the header fields as a group are too large \\
        444 No Response (Nginx) & Used in Nginx logs. Indicates that the server closed the connection without sending any response whatsoever \\
        449 Retry With (Microsoft) & The request should be retried after performing some action \\
        450 Blocked by Windows Parental Controls (Microsoft) & Windows Parental Controls are turned on and are blocking access to the resource \\
        451 Unavailable For Legal Reasons (Internet Draft) & Intended to be used when a resource is being censored or blocked; a reference to Fahrenheit 451 \\
      \end{tabular}
    \end{block}
    \begin{block}{\huge 5XX Server Error}

      \vspace{0.3in}

      \rowcolors{3}{light-gray}{white}
      \begin{tabular}{p{0.3\textwidth} p{0.65\textwidth}}
        Status Code & Description \\ \hline
        500 Internal Server Error & A generic server error message, for when no other more specific message is suitable \\
        501 Not Implemented & The server can not process the request method \\
        502 Bad Gateway & The server is a gateway or proxy, and received a bad response from the upstream server (such as a socket hangup) \\
        503 Service Unavailable & The resource is temporarily unavailable, usually because it is overloaded or down for maintenance \\
        504 Gateway Timeout & The server is a gateway or proxy, and the upstream server did not respond in a reasonable timeframe \\
        505 HTTP Version Not Supported & The server does not support the request's specified HTTP version \\
        507 Insufficient Storage (WebDAV) & The server is out of storage space and can not complete the request \\
        508 Loop Detected (WebDAV) & The server has detected an infinite loop while processing the request \\
        509 Bandwidth Limit Exceeded & A convention used to report that bandwidth limits have been exceeded, and not part of any RFC or spec. \\
      \end{tabular}
    \end{block}
  \end{frame}
\end{document}
